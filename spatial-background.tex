\documentclass[12pt]{article}
\usepackage[T1]{fontenc}

\usepackage{natbib}

\usepackage{etoolbox}
\usepackage{lmodern} \normalfont %to load T1lmr.fd
\usepackage{anyfontsize}
\DeclareFontShape{T1}{lmr}{bx}{sc} { <-> ssub * cmr/bx/sc }{}
\DeclareFontShape{T1}{lmr}{m}{scit}{ <-> ssub * cmr/m/sc }{}
\DeclareFontShape{T1}{lmr}{bx}{scit}{ <-> ssub * cmr/bx/sc }{}

% \include{preamble}

\usepackage{xparse}
\usepackage{xcolor}
\usepackage{soul}
\usepackage{xspace}

\usepackage{amssymb,amsmath}

\newcommand{\plotwidth}{6in}
\newcommand{\plotheight}{3in}
\usepackage{graphicx}


%%% Macros %%%

% Numbers
\newcommand{\Nats}{\mathbb{N}}
\newcommand{\Reals}{\mathbb{R}}

% Highlighting
\newcommand{\mb}[1]{\boldsymbol{#1}}

% Linear algebra
\newcommand{\Tr}{^{\textrm{T}}}

% Math
\newcommand{\norm}[1]{\left\lVert#1\right\rVert}

% Parameters
\newcommand{\params}{\mb{\theta}}

% Data
\newcommand{\response}{\mb{y}}

%%% END Macros %%%

\makeatletter
\ExplSyntaxOn
\cs_new:Npn \white_text:n #1
  {
    \fp_set:Nn \l_tmpa_fp {#1 * .01}
    \llap{\textcolor{white}{\the\SOUL@syllable}\hspace{\fp_to_decimal:N \l_tmpa_fp em}}
    \llap{\textcolor{white}{\the\SOUL@syllable}\hspace{-\fp_to_decimal:N \l_tmpa_fp em}}
  }
\NewDocumentCommand{\whiten}{ m }
    {
      \int_step_function:nnnN {1}{1}{#1} \white_text:n
    }
\ExplSyntaxOff

\NewDocumentCommand{ \varul }{ D<>{5} O{0.2ex} O{0.1ex} +m } {%
\begingroup
\setul{#2}{#3}%
\def\SOUL@uleverysyllable{%
   \setbox0=\hbox{\the\SOUL@syllable}%
   \ifdim\dp0>\z@
      \SOUL@ulunderline{\phantom{\the\SOUL@syllable}}%
      \whiten{#1}%
      \llap{%
        \the\SOUL@syllable
        \SOUL@setkern\SOUL@charkern
      }%
   \else
       \SOUL@ulunderline{%
         \the\SOUL@syllable
         \SOUL@setkern\SOUL@charkern
       }%
   \fi}%
    \ul{#4}%
\endgroup
}
\makeatother

\newcommand{\intexthighlight}[1]{\varul<10>[0.2ex][0.1ex]{#1}}

\title{Spatial models: background and open problems}
\author{}
\date{}

\setlength{\parindent}{0pt}
\setlength{\parskip}{6pt}
\usepackage{titlesec}
\titlespacing{\section}{0pt}{\parskip}{0pt}
\titlespacing{\subsection}{0pt}{\parskip}{0pt}
\titlespacing{\subsubsection}{0pt}{\parskip}{0pt}
\usepackage[letterpaper, margin=1in]{geometry}
% \usepackage[hide=false,setmargin=true,marginparwidth=0.75in]{marginalia}


\begin{document}

\maketitle
\tableofcontents

\newpage

\section{Example: binomial geostatistical regression}

\subsection{Model}

Several authors \citep{loaloa,geostatsp,prevmap} have considered a dataset containing 
counts of subjects who tested positive for a tropical
disease caused by the \emph{Loa loa} parasitic roundworm in $n = 190$
villages in Cameroon and Nigeria. The goal is to model the spatial variation in
the probability of infection. For village $i\in\{1,\ldots,n\}$ let $0\leq Y_{i}\leq N_{i}\in\Nats$ denote the number 
of infected and total number of residents, and $\mb{s}_{i}\in\Reals^{2}$ denote its spatial location.
Let $p(\mb{s})$ denote the probability of infection at location $\mb{s}\in\Reals^{2}$. The model is:
\begin{equation}\begin{aligned}
Y_{i} | p(\mb{s}_{i}) &\overset{ind}{\sim}\text{Binomial}\{N_{i},p(\mb{s}_{i})\}, \\
\log\left\{\frac{p(\mb{s})}{1-p(\mb{s})}\right\} &= \beta_{0} + \mb{x}(\mb{s})\Tr\mb{\beta} + u(\boldsymbol{s}), \mb{s}\in\Reals^{2}, \\
u(\cdot) | \boldsymbol{\mb{\xi}} &\sim \mathcal{GP}(0,\text{C}_{\boldsymbol{\xi}}). \\
\end{aligned}\end{equation} 
The spatially-referenced covariates $\mb{x}(\mb{s}) = (x_1(\mb{s}),\ldots,x_d(\mb{s}))\Tr$ represent continuous elevation and an index of vegetation, and a categorical land type.
The unknown process \(u(\cdot)\)
governs excess spatial variation in logit infection risk, and is
modelled as a Gaussian Process with Matern
covariance function $\text{Cov}\{u(\mb{s}+\mb{h}),u(\mb{s})\} = \text{C}_{\boldsymbol{\xi}}(\norm{\mb{h}}), \mb{s},\mb{h}\in\Reals^{2}$
depending on parameters $\boldsymbol{\xi} = (\sigma,\rho)\Tr$. 
These are interpreted physically as the ``marginal standard deviation'' and ``practical correlation range''; see \citet{geostatsp}.
The primary goal is to make point and interval predictions of $u(\cdot)$ having observed $\mb{Y} = (Y_1,\ldots,Y_n)\Tr$.
This ``excess spatial variation'' is interpreted as the spatial variability in the probability of disease that remains unexplained by the available covariates and can guide investigators/policy makers about where to target resources for prevention, for example.

\subsection{Data}

Figure \ref{fig:loaloapoint} shows the locations and observed prevalences $Y_i/N_i$ of the \texttt{loaloa} data.
Let $A\subset\Reals^2$ represent the longitude (x) and latitude (y) region plotted in Figure \ref{fig:loaloapoint}.
The goal of model fitting is to produce maps of $p(\mb{s})$ and $u(\mb{s})$ for any $\mb{s}\in A$. 

Figures \ref{fig:loaloaelevation}, \ref{fig:loaloaevi}, and \ref{fig:loaloalt} show the available covariates.

\begin{figure}
  \centering
  \includegraphics[width = \plotwidth, height = \plotheight]{figures/loaloa-points.pdf}
  \caption{Loaloa prevalence in Nigeria and Cameroon. Points show locations of villages, sizes are proportional to observed prevalance rate $Y_i/N_i$.}
  \label{fig:loaloapoint}
\end{figure}

\begin{figure}
  \centering
  \includegraphics[width = \plotwidth, height = \plotheight]{figures/loaloa-elevation.pdf}
  \caption{Elevation covariate.}
  \label{fig:loaloaelevation}
\end{figure}

\begin{figure}
  \centering
  \includegraphics[width = \plotwidth, height = \plotheight]{figures/loaloa-evi.pdf}
  \caption{Vegetation index covariate.}
  \label{fig:loaloaevi}
\end{figure}

\begin{figure}
  \centering
  \includegraphics[width = \plotwidth, height = \plotheight]{figures/loaloa-lt.pdf}
  \caption{Land type covariate.}
  \label{fig:loaloalt}
\end{figure}

\subsection{Results}

The model can be fit using the R-INLA package. The easiest way to do this without access to any of the underlying details
is to use Patrick Brown's \texttt{geostatsp} package; see the \texttt{R} file.
This produces estimates $\widehat{u}(\cdot)$ of $u(\cdot)$ and hence $\widehat{p}(\cdot)$ of $p(\cdot)$;
technically it returns posterior modes/means and $2.5\%$ and $97.5\%$ percentiles.
The \texttt{geostatsp::glgm} function summarizes these on a grid and saves them as rasters which can be plotted.
Figures \ref{fig:loaloaumode} -- \ref{fig:loaloaplower} show the results.

\begin{figure}
  \centering
  \includegraphics[width = \plotwidth, height = \plotheight]{figures/loaloa-u-mode.pdf}
  \caption{Posterior mode $\widehat{u}(\cdot)$ of the excess spatial variation $u(\cdot)$.}
  \label{fig:loaloaumode}
\end{figure}

\begin{figure}
  \centering
  \includegraphics[width = \plotwidth, height = \plotheight]{figures/loaloa-u-upperquant.pdf}
  \caption{Posterior $97.5\%$ile $\widehat{u}(\cdot)$ of the excess spatial variation $u(\cdot)$.}
  \label{fig:loaloauupper}
\end{figure}

\begin{figure}
  \centering
  \includegraphics[width = \plotwidth, height = \plotheight]{figures/loaloa-u-lowerquant.pdf}
  \caption{Posterior $2.5\%$ile $\widehat{u}(\cdot)$ of the excess spatial variation $u(\cdot)$.}
  \label{fig:loaloaulower}
\end{figure}

\begin{figure}
  \centering
  \includegraphics[width = \plotwidth, height = \plotheight]{figures/loaloa-p-mode.pdf}
  \caption{Posterior mode $\widehat{p}(\cdot)$ of the predicted prevalence $p(\cdot)$.}
  \label{fig:loaloapmode}
\end{figure}

\begin{figure}
  \centering
  \includegraphics[width = \plotwidth, height = \plotheight]{figures/loaloa-p-upperquant.pdf}
  \caption{Posterior $97.5\%$ile $\widehat{p}(\cdot)$ of the predicted prevalence $p(\cdot)$.}
  \label{fig:loaloapupper}
\end{figure}

\begin{figure}
  \centering
  \includegraphics[width = \plotwidth, height = \plotheight]{figures/loaloa-p-lowerquant.pdf}
  \caption{Posterior $2.5\%$ile $\widehat{p}(\cdot)$ of the predicted prevalence $p(\cdot)$.}
  \label{fig:loaloaplower}
\end{figure}

\section{Details}

\subsection{Background}

This type of analysis is performed and reported all the time with about this much detail.
Here are the details. Let $\params = (\mb{\beta}\Tr, \mb{\xi}\Tr)\Tr$.
In principle, we seek the posterior distribution of $u(\cdot)|\response$.
This is going to require two main ingredients:
\begin{enumerate}
  \item A \textbf{discretization} $\mb{\omega}\in\Reals^m, m\in\Nats$ of the continuous stochastic process $u(\cdot)$ such that for 
  any $\mb{s}\in\Reals^2$ there exists some $j\in\{1,\ldots,m\}$ such that $\omega_j \approx u(\mb{s})$ in some sense.
  \item An \textbf{approximation} to the (intractable) posterior distribution of $\mb{\omega}|\response$ where $\response = (y_1,\ldots,y_n)\Tr\in\Reals^n$ are the observed data.
\end{enumerate}
The details are given below. \textbf{One open problem is to determine the asymptotic approximation error in} $u(\cdot)|\response$ \textbf{as a function of} $m$ \textbf{and} $n$, although \citet{sanz2022finite} provides one perspective.
The manner in which I currently see this being done is by extending the analyses of \citet{bilodeau2024stochastic,stringer2022asymptotics}
such that the regularity conditions accommodate spatial models using the ``SPDE'' discretization of \citet{spde,sanz2022finite}.

\subsection{Using the observed $\mb{s}$ values}

Recall that $u(\cdot)$ is a zero-mean stationary Gaussian process, so it has the property that for any realization $\mb{s}_1,\ldots,\mb{s}_m, m\in\Nats$,
$(u(\mb{s}_1),\ldots,u(\mb{s}_m))\Tr$ jointly follows a Gaussian distribution with covariance matrix $\Sigma$ determined by $\Sigma_{ij} = C_{\mb{\xi}}(\norm{\mb{s}_{i} - \mb{s}_{j}})$.
Since $u(\cdot)$ is a continuous stochastic process, in practice (i.e. when analyzing data on a computer) we do not work with its
posterior distribution, but rather a posterior distribution of some suitable discretization of the process.
The most obvious choice is to infer the posterior distribution of
$$
\mb{\omega}_n = (u(\mb{s}_1),\ldots,u(\mb{s}_n))\Tr,
$$
where $\mb{s}_1,\ldots,\mb{s}_n$ are the \emph{observed} locations, i.e. the points plotted in Figure \ref{fig:loaloapoint}.
This ``discretization'' is conceptually straightforward but leads to a few challenges that render it practically infeasible:
\begin{enumerate}
  \item \textbf{Computational burden}: the precision (inverse covariance) matrix of $(u(\mb{s}_1),\ldots,u(\mb{s}_n))|\response$ in this 
  setup is dense (all values may be nonzero), has no other known exploitable structure, and has dimension $n\times n$, leading to a memory cost that scales as $O(n^2)$ to store it and a computational cost (number of floating point operations) that scales as $O(n^3)$ to decompose it; this renders inference inconvenient for $n$ beyond a few hundred and impossible for $n$ beyond a few thousand even on big computers.
  \item \textbf{Numerical instability}: as $n$ increases but the spatial domain $A\subset\Reals^2$ stays fixed, the $\mb{s}_i$ will tend
  to be close to each other. Consequently, while the rank of $\Sigma$ stays equal to $n$ mathematically/theoretically, numerically the matrix $\Sigma$ becomes ill-conditioned (condition number---ratio of smallest to largest singular values---very small), leading to large finite-precision arithmetic errors in the output of the numerical linear algebra algorithms used for inference. This is actually the bigger problem; in practice, for ``usual'' datasets, this starts to give bad results/crash for $n$ in the hundreds.
\end{enumerate}
To mitigate these challenges, the current state-of-the-art approach is to use the stochastic partial differential equation (SPDE) approach to discretization of $u(\cdot)$.

\subsection{SPDE Discretization}\label{subsec:spde}

A Gaussian process is defined by its covariance function.
The Matern covariance is almost always used in geospatial/geostatistical models; see \citet[Appendix A]{geostatsp} for explicit details.
Where does the Matern covariance come from?
It can be shown (not by me, but maybe by you!) that the solution, $u(\cdot)$, to the SPDE
\begin{equation}\label{eqn:maternspde}
(\kappa^2 - \Delta)^{\alpha/2}u(\mb{s}) = W(\mb{s}), \alpha = \nu + d/2, d = \text{dim}(\mb{s}) = 2, \kappa > 0, \nu > 0,
\end{equation}
is a Gaussian process with Matern covariance function. 
Here $W(\cdot)$ is white noise in $\Reals^d$, $d=2$, and
$$
\Delta = \sum_{i=1}^{d}\frac{\partial^2}{\partial s_i^2}
$$
is the Laplacian.
See \citet[Eq. (1), (2), and (4)]{spde} and note that there might be
slight differences between \citet{spde} and \citet{geostatsp}.

Since the exact solution to Eq. (\ref{eqn:maternspde}) has been known since the 1960's, the relevance of this characterization
to geostatistical practice was not appreciated for some time; everyone just used the formula for the covariance function that solves (\ref{eqn:maternspde}).
\citet{spde}, however, made the controversial suggestion that---despite knowing the \emph{exact} answer to (\ref{eqn:maternspde})---
computations be based on an \emph{approximation} to it. To this end, the finite element method/Galerkin approximation is applied
to approximate the solution to (\ref{eqn:maternspde}); see \citet{spde} for the original construction, \citet{miller2020understanding}
for a (very) clear explanation, and \citet{zhang2024model} for a recent application by a PhD student at Toronto with me and our mutual supervisors. 

The relevant details are as follows. Define a partition of $A\subset\Reals^2$ with $m\in\Nats$ regions $A_{j},j=1,\ldots,m$.
This can be a rectangular grid, but triangles are preferred by the numerical analysis folks; see \citet{simpson2016going} for relevance to geostatistics.
A discretization $\mb{\omega} = (\omega_1,\ldots,\omega_m)\Tr$ of $u(\cdot)$ onto $A_1,\ldots,A_m$ is given by applying the finite element method to approximate (\ref{eqn:maternspde}) n $A_1,\ldots,A_m$.
Since the $A_j$ partition $A$, for any $\mb{s}\in\Reals^2$ there exists some $j$ such that $u(\mb{s})\approx\omega_j$.
(The specific meaning of ``$\approx$'' depends on the convergence perspective taken.)
Inferences about $u(\cdot)$ are made by making inferences about $\mb{\omega}$ and then replacing $u(\cdot)$ by $\mb{\omega}$
in all reported summaries. Figures \ref{fig:loaloaumode} -- \ref{fig:loaloaulower} show posterior summaries of $\mb{\omega}$
for the \texttt{Loaloa} example.

This approach addresses both challenges:
\begin{enumerate}
  \item The mechanics of the approximation guarantee that the precision matrix of $\omega_1,\ldots,\omega_m$ is \emph{sparse}, meaning only $O(m)$ of its $n^2$ values may be nonzero.
  This reduces the computational complexity of storage and decomposition to $O(m)$ from $O(n^3)$, a huge reduction and more importantly one that is user-controlled–- the user can pick the $m$ that their computer can handle, explicitly trading accuracy for speed.
  \item The partition $A_1,\ldots,A_m$ can be formed in such a manner as to avoid numerical instability, e.g. by keeping the distances between the centroids of the $A_j$ relatively uniform; see \citet{simpson2016going} for a treatment.
\end{enumerate}

There is numerical convergence theory about the accuracy of the discretization as $m\to\infty$; see \citet{spde}.
To my knowledge, there is not much satistical convergence theory about the accuracy of the posterior $\mb{\omega}|\response$
as an approximation to the posterior $u(\cdot)|\response$; \citet{sanz2022finite} is the only paper I've seen.

\subsection{Posterior approximation: Laplace approximation}\label{subsec:laplace}

I use $\pi(\cdot)$ and $\pi(\cdot|\cdot)$ to represent marginal and conditional densities.
The posterior density of $\mb{\omega}|\response$ is:
\begin{equation}\label{eqn:posterior}
\pi(\mb{\omega}|\response) = \frac{\pi(\response|\mb{\omega})\pi(\mb{\omega})}{\pi(\response)}.
\end{equation}
The numerator is tractable: $\pi(\mb{\omega})$ is Gaussian with covariance matrix determined by the SPDE approximation;
$\pi(\response|\mb{\omega})$ is the likelihood, Binomial in the \texttt{loaloa} example.
The denominator,
\begin{equation}\label{eqn:normconst}
\pi(\response) = \int\pi(\response|\mb{\omega})\pi(\mb{\omega})d\mb{\omega},
\end{equation}
is an intractable integral of dimension $m$, which is generally moderate (dozens) to large (hundreds -- thousands).
Further, while I'm not sure the degree to which this has been formally worked out, it will plausibly be the case 
that $m$ will have to increase with $n$ at some rate in order for the discretization error to go to zero-- this is the
case with spline models \citep{kauermann2009some} which are very closely related to spatial models and SPDE discretization \citep{miller2020understanding}. So, we can regard (\ref{eqn:normconst}) as being a \emph{high-dimensional} integral, in the specific sense
that it is an integral over a parameter whose dimension $m$, increases as more data, $n$, are obtained. This dramatically
complicates convergence theory for approximations to this integral.

Monte Carlo methods for (\ref{eqn:normconst}) are well studied in spatial problems, and are generally disfavoured for being slow
and challenging to implement well. The status quo is to use the Laplace approximation to (\ref{eqn:normconst}).
Let $\ell(\mb{\omega};\response) = -\log\pi(\response|\mb{\omega})\pi(\mb{\omega})$ so $\pi(\response) = \int\exp\left\{-\ell(\mb{\omega};\response)\right\}d\mb{\omega}$.
Let $\widehat{\mb{\omega}} = \text{argmin}_{\mb{\omega}}\ell(\mb{\omega};\response)$ and $\widehat{\mb{H}} = \partial^2_{\mb{\omega}\mb{\omega}\Tr}\ell(\widehat{\mb{\omega}};\response)$.
The Laplace approximation to $\pi(\response)$ is
\begin{equation}\label{eqn:laplace}
\widetilde{\pi}(\response) = (2\pi)^{m/2}\lvert\widehat{\mb{H}}\rvert^{-1/2}\exp\left\{-\ell(\widehat{\mb{\omega}};\response)\right\}.
\end{equation}
The approximation is obtained via second-order Taylor expansion of $\ell(\mb{\omega};\response)$ at $\widehat{\mb{\omega}}$.
Under ``classical'' regularity conditions (which exclude $m$ increasing with $n$ and exclude spatial models) the error is $O_p(n^{-1/2})$
independent of $m$; see \citet[Theorem 1]{bilodeau2024stochastic} for the precise statement and the appendix of that paper for precise
conditions. Work on the high dimensional case includes \citet{shun1995laplace} and \citet{tang2024highdim}.

To my knowledge, it would be a \textbf{novel contribution} to come up with regularity conditions that include spatial SPDE models
under which Theorem 1 of \citet{bilodeau2024stochastic} can be applied to convergence of (\ref{eqn:laplace}).

\subsection{Posterior approximation: integrated nested Laplace approximation}\label{subsec:inla}

There are two further practical considerations not covered by the basic Laplace approximation:
\begin{enumerate}
  \item We also have unknown parameters, $\params$, that either need to be inferred or at least accounted for,
  \item A high-dimensional joint posterior is not a useful object on its own; what we actually want are marginal posterior summaries, such as moments and quantiles of $\omega_j$ for any $j$. This requires further approximation(s).
\end{enumerate}
Neither problem occurs when using Monte Carlo methods to sample from posteriors. If you had a sample from the joint posterior of $(\mb{\omega},\params)|\response$, you can just use sample estimates of any desired posterior summary. That this is even a challenge worth mentioning
is unique to the use of non-sample-based approximate Bayesian inference methods.

The Integrated Nested Laplace Approximation (INLA) was introduced in the seminal work of \citet{rue2009approximate}.
The explanation there is quite complicated with computational and methodological details intermixed. Ignoring the computational
details (which were the primary contribution of the work), the basic approximations aren't too hard to write down.
The parameters are now $(\mb{\omega},\params)$ with posterior
$$
\pi(\mb{\omega},\params|\response) = \frac{\pi(\response|\mb{\omega},\params)\pi(\mb{\omega}|\params)\pi(\params)}{\pi(\response)},
$$
and interest lies in the \emph{marginal} posteriors
\begin{align}\label{eqn:margpost}
\pi(\omega_j | \response) &= \int\pi(\mb{\omega},\params|\response)d\mb{\omega}_{-j}d\params, j = 1,\ldots,m,\\
\pi(\theta_l | \response) &= \int\pi(\mb{\omega},\params|\response)d\mb{\omega}d\params_{-l}, l = 1,\ldots,p=\text{dim}(\params). \\
\end{align}
All of these integrals are also intractable and require further approximation. The INLA method gives a computationally feasible
way to compute the Laplace approximations to $\pi(\omega_j | \response)$ (we'll ignore $\pi(\theta_l | \response)$) for now.
Once an approxiation to $\pi(\omega_j | \response)$ is obtained, summaries such as means and quantiles can be easily approximated
to high numerical accuracy using any integration rule, e.g. the trapezoid rule; it is not expected that this final approximation step 
will introduce meaningful error. Indeed, see \citet{bilodeau2024stochastic} for a discussion with details in the supplement.

INLA is used widely for spatial statistics, with around 10 books, dozens of followup papers, thousands of citations, and countless
web tutorials, workshops, and so forth dedicated to its use. \textbf{Nobody has formally analyzed its convergence behaviour}.
In my thesis work \citep{stringer2023fast} I provide a theorem that applies only to a simplified version of INLA and under very
restrictive conditions that definitely rule out spatial models.

\section{Open problem(s)}

Several open problems are identified from the above; these will need to be refined and the literature investigated to crystalize
them into a proposed thesis for your NSERC application:
\begin{enumerate}
  \item Convergence of the SPDE posterior approximation (\S\ref{subsec:spde}),
  \item Convergence of the Laplace approximation to the posterior of the SPDE discretization (\S\ref{subsec:laplace}),
  \item Convergence of INLA for spatial SPDE models (\S\ref{subsec:inla}).
\end{enumerate}
Overall, the quantifying the error in using an approximation to $\pi(\mb{\omega}|\response)$ or $\pi(\omega_j|\response)$ as an approximation of $\pi(u(\cdot)|\response)$ in terms of $m$ and $n$ would be a substantial contribution to the theory of geostatistical practice.


\bibliographystyle{apalike}
\bibliography{spatial-background.bib}

\end{document}